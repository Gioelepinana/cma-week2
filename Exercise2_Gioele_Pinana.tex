% Options for packages loaded elsewhere
\PassOptionsToPackage{unicode}{hyperref}
\PassOptionsToPackage{hyphens}{url}
%
\documentclass[
]{article}
\usepackage{amsmath,amssymb}
\usepackage{lmodern}
\usepackage{iftex}
\ifPDFTeX
  \usepackage[T1]{fontenc}
  \usepackage[utf8]{inputenc}
  \usepackage{textcomp} % provide euro and other symbols
\else % if luatex or xetex
  \usepackage{unicode-math}
  \defaultfontfeatures{Scale=MatchLowercase}
  \defaultfontfeatures[\rmfamily]{Ligatures=TeX,Scale=1}
\fi
% Use upquote if available, for straight quotes in verbatim environments
\IfFileExists{upquote.sty}{\usepackage{upquote}}{}
\IfFileExists{microtype.sty}{% use microtype if available
  \usepackage[]{microtype}
  \UseMicrotypeSet[protrusion]{basicmath} % disable protrusion for tt fonts
}{}
\makeatletter
\@ifundefined{KOMAClassName}{% if non-KOMA class
  \IfFileExists{parskip.sty}{%
    \usepackage{parskip}
  }{% else
    \setlength{\parindent}{0pt}
    \setlength{\parskip}{6pt plus 2pt minus 1pt}}
}{% if KOMA class
  \KOMAoptions{parskip=half}}
\makeatother
\usepackage{xcolor}
\IfFileExists{xurl.sty}{\usepackage{xurl}}{} % add URL line breaks if available
\IfFileExists{bookmark.sty}{\usepackage{bookmark}}{\usepackage{hyperref}}
\hypersetup{
  pdftitle={PTED / Exercise 2},
  pdfauthor={Gioele Pinana},
  hidelinks,
  pdfcreator={LaTeX via pandoc}}
\urlstyle{same} % disable monospaced font for URLs
\usepackage[margin=1in]{geometry}
\usepackage{color}
\usepackage{fancyvrb}
\newcommand{\VerbBar}{|}
\newcommand{\VERB}{\Verb[commandchars=\\\{\}]}
\DefineVerbatimEnvironment{Highlighting}{Verbatim}{commandchars=\\\{\}}
% Add ',fontsize=\small' for more characters per line
\usepackage{framed}
\definecolor{shadecolor}{RGB}{248,248,248}
\newenvironment{Shaded}{\begin{snugshade}}{\end{snugshade}}
\newcommand{\AlertTok}[1]{\textcolor[rgb]{0.94,0.16,0.16}{#1}}
\newcommand{\AnnotationTok}[1]{\textcolor[rgb]{0.56,0.35,0.01}{\textbf{\textit{#1}}}}
\newcommand{\AttributeTok}[1]{\textcolor[rgb]{0.77,0.63,0.00}{#1}}
\newcommand{\BaseNTok}[1]{\textcolor[rgb]{0.00,0.00,0.81}{#1}}
\newcommand{\BuiltInTok}[1]{#1}
\newcommand{\CharTok}[1]{\textcolor[rgb]{0.31,0.60,0.02}{#1}}
\newcommand{\CommentTok}[1]{\textcolor[rgb]{0.56,0.35,0.01}{\textit{#1}}}
\newcommand{\CommentVarTok}[1]{\textcolor[rgb]{0.56,0.35,0.01}{\textbf{\textit{#1}}}}
\newcommand{\ConstantTok}[1]{\textcolor[rgb]{0.00,0.00,0.00}{#1}}
\newcommand{\ControlFlowTok}[1]{\textcolor[rgb]{0.13,0.29,0.53}{\textbf{#1}}}
\newcommand{\DataTypeTok}[1]{\textcolor[rgb]{0.13,0.29,0.53}{#1}}
\newcommand{\DecValTok}[1]{\textcolor[rgb]{0.00,0.00,0.81}{#1}}
\newcommand{\DocumentationTok}[1]{\textcolor[rgb]{0.56,0.35,0.01}{\textbf{\textit{#1}}}}
\newcommand{\ErrorTok}[1]{\textcolor[rgb]{0.64,0.00,0.00}{\textbf{#1}}}
\newcommand{\ExtensionTok}[1]{#1}
\newcommand{\FloatTok}[1]{\textcolor[rgb]{0.00,0.00,0.81}{#1}}
\newcommand{\FunctionTok}[1]{\textcolor[rgb]{0.00,0.00,0.00}{#1}}
\newcommand{\ImportTok}[1]{#1}
\newcommand{\InformationTok}[1]{\textcolor[rgb]{0.56,0.35,0.01}{\textbf{\textit{#1}}}}
\newcommand{\KeywordTok}[1]{\textcolor[rgb]{0.13,0.29,0.53}{\textbf{#1}}}
\newcommand{\NormalTok}[1]{#1}
\newcommand{\OperatorTok}[1]{\textcolor[rgb]{0.81,0.36,0.00}{\textbf{#1}}}
\newcommand{\OtherTok}[1]{\textcolor[rgb]{0.56,0.35,0.01}{#1}}
\newcommand{\PreprocessorTok}[1]{\textcolor[rgb]{0.56,0.35,0.01}{\textit{#1}}}
\newcommand{\RegionMarkerTok}[1]{#1}
\newcommand{\SpecialCharTok}[1]{\textcolor[rgb]{0.00,0.00,0.00}{#1}}
\newcommand{\SpecialStringTok}[1]{\textcolor[rgb]{0.31,0.60,0.02}{#1}}
\newcommand{\StringTok}[1]{\textcolor[rgb]{0.31,0.60,0.02}{#1}}
\newcommand{\VariableTok}[1]{\textcolor[rgb]{0.00,0.00,0.00}{#1}}
\newcommand{\VerbatimStringTok}[1]{\textcolor[rgb]{0.31,0.60,0.02}{#1}}
\newcommand{\WarningTok}[1]{\textcolor[rgb]{0.56,0.35,0.01}{\textbf{\textit{#1}}}}
\usepackage{graphicx}
\makeatletter
\def\maxwidth{\ifdim\Gin@nat@width>\linewidth\linewidth\else\Gin@nat@width\fi}
\def\maxheight{\ifdim\Gin@nat@height>\textheight\textheight\else\Gin@nat@height\fi}
\makeatother
% Scale images if necessary, so that they will not overflow the page
% margins by default, and it is still possible to overwrite the defaults
% using explicit options in \includegraphics[width, height, ...]{}
\setkeys{Gin}{width=\maxwidth,height=\maxheight,keepaspectratio}
% Set default figure placement to htbp
\makeatletter
\def\fps@figure{htbp}
\makeatother
\setlength{\emergencystretch}{3em} % prevent overfull lines
\providecommand{\tightlist}{%
  \setlength{\itemsep}{0pt}\setlength{\parskip}{0pt}}
\setcounter{secnumdepth}{-\maxdimen} % remove section numbering
\ifLuaTeX
  \usepackage{selnolig}  % disable illegal ligatures
\fi

\title{PTED / Exercise 2}
\author{Gioele Pinana}
\date{2022-04-29}

\begin{document}
\maketitle

\hypertarget{task-1-import-your-data}{%
\subsection{Task 1: Import your Data}\label{task-1-import-your-data}}

Load the necessary libraries

\begin{Shaded}
\begin{Highlighting}[]
\FunctionTok{library}\NormalTok{(readr)        }\CommentTok{\# to import tabular data (e.g. csv)}
\FunctionTok{library}\NormalTok{(dplyr)        }\CommentTok{\# to manipulate (tabular) data}
\end{Highlighting}
\end{Shaded}

\begin{verbatim}
## 
## Attache Paket: 'dplyr'
\end{verbatim}

\begin{verbatim}
## Die folgenden Objekte sind maskiert von 'package:stats':
## 
##     filter, lag
\end{verbatim}

\begin{verbatim}
## Die folgenden Objekte sind maskiert von 'package:base':
## 
##     intersect, setdiff, setequal, union
\end{verbatim}

\begin{Shaded}
\begin{Highlighting}[]
\FunctionTok{library}\NormalTok{(ggplot2)      }\CommentTok{\# to visualize data}
\FunctionTok{library}\NormalTok{(sf)           }\CommentTok{\# to handle spatial vector data}
\end{Highlighting}
\end{Shaded}

\begin{verbatim}
## Linking to GEOS 3.9.1, GDAL 3.2.1, PROJ 7.2.1; sf_use_s2() is TRUE
\end{verbatim}

\begin{Shaded}
\begin{Highlighting}[]
\FunctionTok{library}\NormalTok{(terra)        }\CommentTok{\# To handle raster data}
\end{Highlighting}
\end{Shaded}

\begin{verbatim}
## terra 1.5.21
\end{verbatim}

\begin{verbatim}
## 
## Attache Paket: 'terra'
\end{verbatim}

\begin{verbatim}
## Das folgende Objekt ist maskiert 'package:ggplot2':
## 
##     arrow
\end{verbatim}

\begin{verbatim}
## Das folgende Objekt ist maskiert 'package:dplyr':
## 
##     src
\end{verbatim}

\begin{Shaded}
\begin{Highlighting}[]
\FunctionTok{library}\NormalTok{(lubridate)    }\CommentTok{\# To handle dates and times}
\end{Highlighting}
\end{Shaded}

\begin{verbatim}
## 
## Attache Paket: 'lubridate'
\end{verbatim}

\begin{verbatim}
## Die folgenden Objekte sind maskiert von 'package:terra':
## 
##     intersect, union
\end{verbatim}

\begin{verbatim}
## Die folgenden Objekte sind maskiert von 'package:base':
## 
##     date, intersect, setdiff, union
\end{verbatim}

Import the downloaded csv

\begin{Shaded}
\begin{Highlighting}[]
\NormalTok{wildschwein\_BE }\OtherTok{\textless{}{-}} \FunctionTok{read\_delim}\NormalTok{(}\StringTok{"wildschwein\_BE\_2056.csv"}\NormalTok{,}\StringTok{","}\NormalTok{) }\CommentTok{\# adjust path}
\end{Highlighting}
\end{Shaded}

\begin{verbatim}
## Rows: 51246 Columns: 6
## -- Column specification --------------------------------------------------------
## Delimiter: ","
## chr  (2): TierID, TierName
## dbl  (3): CollarID, E, N
## dttm (1): DatetimeUTC
## 
## i Use `spec()` to retrieve the full column specification for this data.
## i Specify the column types or set `show_col_types = FALSE` to quiet this message.
\end{verbatim}

\begin{Shaded}
\begin{Highlighting}[]
\NormalTok{wildschwein\_BE }\OtherTok{\textless{}{-}} \FunctionTok{st\_as\_sf}\NormalTok{(wildschwein\_BE, }\AttributeTok{coords =} \FunctionTok{c}\NormalTok{(}\StringTok{"E"}\NormalTok{, }\StringTok{"N"}\NormalTok{), }\AttributeTok{crs =} \DecValTok{2056}\NormalTok{, }\AttributeTok{remove =} \ConstantTok{FALSE}\NormalTok{)}
\end{Highlighting}
\end{Shaded}

\hypertarget{task-2-getting-an-overview}{%
\subsection{Task 2: Getting an
Overview}\label{task-2-getting-an-overview}}

\begin{Shaded}
\begin{Highlighting}[]
\NormalTok{wildschwein\_BE}\SpecialCharTok{$}\NormalTok{timelag  }\OtherTok{\textless{}{-}} \FunctionTok{as.integer}\NormalTok{(}\FunctionTok{difftime}\NormalTok{(}\FunctionTok{lead}\NormalTok{(wildschwein\_BE}\SpecialCharTok{$}\NormalTok{DatetimeUTC), wildschwein\_BE}\SpecialCharTok{$}\NormalTok{DatetimeUTC, }\AttributeTok{units =} \FunctionTok{c}\NormalTok{(}\StringTok{"secs"}\NormalTok{)))}
\NormalTok{wildschwein\_BE}
\end{Highlighting}
\end{Shaded}

\begin{verbatim}
## Simple feature collection with 51246 features and 7 fields
## Geometry type: POINT
## Dimension:     XY
## Bounding box:  xmin: 2568153 ymin: 1202306 xmax: 2575154 ymax: 1207609
## Projected CRS: CH1903+ / LV95
## # A tibble: 51,246 x 8
##    TierID TierName CollarID DatetimeUTC                E        N
##  * <chr>  <chr>       <dbl> <dttm>                 <dbl>    <dbl>
##  1 002A   Sabi        12275 2014-08-22 21:00:12 2570409. 1204752.
##  2 002A   Sabi        12275 2014-08-22 21:15:16 2570402. 1204863.
##  3 002A   Sabi        12275 2014-08-22 21:30:43 2570394. 1204826.
##  4 002A   Sabi        12275 2014-08-22 21:46:07 2570379. 1204817.
##  5 002A   Sabi        12275 2014-08-22 22:00:22 2570390. 1204818.
##  6 002A   Sabi        12275 2014-08-22 22:15:10 2570390. 1204825.
##  7 002A   Sabi        12275 2014-08-22 22:30:13 2570387. 1204831.
##  8 002A   Sabi        12275 2014-08-22 22:45:11 2570381. 1204840.
##  9 002A   Sabi        12275 2014-08-22 23:00:27 2570316. 1204935.
## 10 002A   Sabi        12275 2014-08-22 23:15:41 2570393. 1204815.
## # ... with 51,236 more rows, and 2 more variables: geometry <POINT [m]>,
## #   timelag <int>
\end{verbatim}

\end{document}
